\documentclass[
  leqno, 
  theme = Fresh Green,
  font = source-han,
  lang = en,
  % mode = simple,
  usesamecnt
]{elegantbook-l3}

\usepackage{amsmath, amssymb, array}

\newfontfamily\scfont{GoSmallCaps.ttf}[
  AutoFakeBold = 2.0,
  SmallCapsFont = *
]

\ElegantCoverInfo{
  title = {\scfont Principles Of  \\[-5pt] \scfont Mathematical Analysis},
  subtitle = {数学分析原理},
  cover image = { cover-3.jpg },
  author = {\scfont Walter Rudin \& Ljguo},
  institute = None
}


\usepackage{tasks}
\usetikzlibrary{decorations.pathmorphing}


\ExplSyntaxOn

\elegant_declare_language:nn { en }
  {
    \RequirePackage[UTF8]{ctex}
    \RequirePackage{bbding}
    \elegant_define_name:n 
      {
        contents = Contents,
        figure = Figure,
        table = Table,
        listoffigure = List Of Figures,
        listoftable = List Of Tables,
        chapter / before  = ,
          chapter / after = ,
          chapter / number = \arabic{chapter},
        section / number = \thesection,
        subsection / number = \thesubsection,
        subsubsection / number = \thesubsubsection,
        theorem = Theorem,
          theorem / icon = $\heartsuit$,
        definition = Definition,
          definition / icon = $\clubsuit$,
        postulate = Postulate,
          postulate / icon = $\Delta$,
        axiom = Axiom,
        corollary = Corollary,
        lemma = Lemma,
          lemma / icon = $\spadesuit$,
        proposition = Proposition,
        remark = Remark,
        introduction = Introduction,
          introduction / icon = \SquareShadowBottomRight
      }
  }
\tl_new:N \g__mark_note_tl 
\tl_gput_right:Nn \g__mark_note_tl { \begin{tasks}[label = \underline{\arabic*}., label-width = 15pt](2) }
\newcounter{notemark}
\NewDocumentCommand{\notemark}{mm}
  {
    % 标注
    \tikz[baseline = (box.base)]{
      \node[inner~sep = 0pt] (box) {#1};
      \draw[purple, thin, overlay, decoration = {snake, amplitude=.3mm,segment~length = 1.5mm, post~length = 0mm}, decorate] ([yshift = -.05\baselineskip]box.south~west) -- ([yshift = -.05\baselineskip]box.south~east);
      \node[inner~sep = 0pt, overlay, yshift = -.28\baselineskip, scale = .75, text = purple, font=\kaishu\bfseries] at (box.south){\refstepcounter{notemark}#2\label{nm:\thenotemark}}
    }
    % 记录
    \tl_gput_right:Ne \g__mark_note_tl 
      { 
        \exp_not:N \task 
        \exp_not:N \textbf{#1}~#2,~  
        \exp_not:N \hyperref[nm:\thenotemark]
          {
            P\exp_not:N \textsuperscript{\thepage} 
          } 
      }
  }

\NewDocumentCommand{\shownotemarks}{}
  {
    \tl_gput_right:Nn \g__mark_note_tl { \end{tasks} }
    \chapter*{\textsc{List~Of~Marks}}
    \addcontentsline{toc}{chapter}{List~Of~Marks}
    \tl_use:N \g__mark_note_tl
  }

% \let\notemark\use_i:nn

\ExplSyntaxOff

\parindent2em

\begin{document}


\maketitle[Skyrmion]

\frontmatter

\tableofcontents

\mainmatter
\linespread{1.6}\selectfont

\chapter{\textsc{The Real And Complex Number Systems}}
\section*{Introduction}
A \notemark{satisfactory discussion}{令人满意的讨论} of the main \notemark{concepts}{概念} of analysis (such as \notemark{convergence}{收敛}, continuity, \notemark{differentiation}{微分}, and integration) must be based on an \notemark{accurately}{精确地} defined number concept. We \notemark{shall}{助动词, 类似于 will} not, however, enter into any discussion of the \notemark{axioms}{公理} that \notemark{govern}{统治, 管理} the arithmetic of the integers, but assume familiarity with the rational numbers (i.e., the numbers of the form $m / n$, where $m$ and $n$ are integers and $n \neq 0$).

The rational number system is \notemark{inadequate}{不足的} for many \notemark{purposes}{目的, 意图}, both as a field and as an ordered set. (These \notemark{terms}{术语} will be defined in Secs. 1.6 and 1.12.) For instance, there is no rational $p$ such that $p^2=2$. (We shall prove this presently.) This leads to the introduction of so-called ``\notemark{irrational numbers}{无理数}" which are often written as \notemark{infinite decimal}{无限小数} expansions and are considered to be ``\notemark{approximated}{近似, 估计}" by the \notemark{corresponding}{对应的} finite decimals. Thus the sequence
\[
  1,1.4,1.41,1.414,1.4142, \ldots
\]
``\notemark{tends to}{趋近于} $\sqrt{2}$." \notemark{But unless}{否则, 表示转折} the irrational number $\sqrt{2}$ has been clearly defined, the question must arise: Just what is it that this sequence ``tends to"?

This sort of question can be answered as soon as the so-called ``real number system" is \notemark{constructed}{建造, 构造}.
\begin{theorem}
  We now show that the equation
  \begin{equation}\label{eq:1.1}
    p^2 = 2,
  \end{equation}
  is not \notemark{satisfied}{满足} by any rational $p$. If there were such a $p$, we could write $p=m / n$ where $m$ and $n$ are integers that are not both even. Let us \notemark{assume}{假设} this is done. Then \eqref{eq:1.1} \notemark{implies}{表明, 意味着}
  \begin{equation}\label{eq:1.2}
    m^2 = 2n^2,
  \end{equation}
  This shows that $m^2$ is even. \notemark{Hence}{因此} $m$ is even (if $m$ \notemark{were}{虚拟语气} odd, $m^2$ would be odd), and so $m^2$ is divisible by 4. It follows that the right side of \eqref{eq:1.2} is \notemark{divisible by}{被~... 整除} 4, so that $n^2$ is even, which implies that $n$ is even.

  The \notemark{assumption}{假定} that \eqref{eq:1.1} holds thus \notemark{leads to}{导致} the conclusion that both $m$ and $n$ are even, \notemark{contrary}{相反的} to our choice of $m$ and $n$. Hence \eqref{eq:1.2} is \notemark{impossible}{不可能的} for rational $p$.

  We now \notemark{examine}{研究} this situation \notemark{a little more closely}{更加仔细地}. Let $A$ be the set of all positive rationals $p$ such that $p^2<2$ and let $B$ \notemark{consist of}{由~...组成} all positive rationals $p$ such that $p^2>2$. We shall show that $A$ contains no largest number and $B$ contains no smallest.

  \notemark{More explicitly}{更加清晰地}, for every $p$ in $A$ we can find a rational $q$ in $A$ such that $p<q$, and for every $p$ in $B$ we can find a rational $q$ in $B$ such that $q<p$. To do this, we \notemark{associate}{关联, 联系} with each rational $p>0$ the number
  \begin{equation}\label{eq:1.3}
    q=p-\frac{p^2-2}{p+2}=\frac{2 p+2}{p+2}
  \end{equation}
  Then
  \begin{equation}\label{eq:1.4}
    q^2-2=\frac{2\left(p^2-2\right)}{(p+2)^2}
  \end{equation}
  
  If $p$ is in $A$ then $p^2-2<0$, \eqref{eq:1.3} \notemark{shows that}{表明} $q>p$, and \eqref{eq:1.4} shows that $q^2<2$. Thus $q$ is in $A$.

  If $p$ is in $B$ then $p^2-2>0$, \eqref{eq:1.3} shows that $0<q<p$, and \eqref{eq:1.4} shows that $q^2>2$. Thus $q$ is in $B$.
\end{theorem}

\begin{remark}
  The \notemark{purpose of}{... 的目标是} the above discussion has been to show that the rational number system has certain gaps, \notemark{in spite of the fact}{尽管事实如此} that between any two rationals there is another: If $r < s$ then $r < (r+s)/2 < s$. The real number system fills these gaps. This is the \notemark{principal}{最重要的, 首要的} reason for the \notemark{fundamental}{根本的, 基本的} role which it plays in analysis. 
\end{remark}

\shownotemarks



\end{document}










