\chapter{\textsc{The Real And Complex Number Systems}}
\section*{Introduction}
A \notemark{satisfactory discussion}{令人满意的讨论} of the main \notemark{concepts}{概念} of analysis (such as \notemark{convergence}{收敛}, continuity, \notemark{differentiation}{微分}, and integration) must be based on an \notemark{accurately}{精确地} defined number concept. We \notemark{shall}{助动词, 类似于 will} not, however, enter into any discussion of the \notemark{axioms}{公理} that \notemark{govern}{统治, 管理} the arithmetic of the integers, but assume familiarity with the rational numbers (i.e., the numbers of the form $m / n$, where $m$ and $n$ are integers and $n \neq 0$).

The rational number system is \notemark{inadequate}{不足的} for many \notemark{purposes}{目的, 意图}, both as a field and as an ordered set. (These \notemark{terms}{术语} will be defined in Secs. 1.6 and 1.12.) For instance, there is no rational $p$ such that $p^2=2$. (We shall prove this presently.) This leads to the introduction of so-called ``\notemark{irrational numbers}{无理数}" which are often written as \notemark{infinite decimal}{无限小数} expansions and are considered to be ``\notemark{approximated}{近似, 估计}" by the \notemark{corresponding}{对应的} finite decimals. Thus the sequence
\[
  1,1.4,1.41,1.414,1.4142, \ldots
\]
``\notemark{tends to}{趋近于} $\sqrt{2}$." \notemark{But unless}{否则, 表示转折} the irrational number $\sqrt{2}$ has been clearly defined, the question must arise: Just what is it that this sequence ``tends to"?

This sort of question can be answered as soon as the so-called ``real number system" is \notemark{constructed}{建造, 构造}.
\begin{theorem}
  We now show that the equation
  \begin{equation}\label{eq:2.1}
    p^2 = 2,
  \end{equation}
  is not \notemarkwavy{satisfied}{满足} by any rational $p$. If there were such a $p$, we could write $p=m / n$ where $m$ and $n$ are integers that are not both even. Let us \notemarkwavy{assume}{假设} this is done. Then \eqref{eq:2.1} \notemarkwavy{implies}{表明, 意味着}
  \begin{equation}\label{eq:2.2}
    m^2 = 2n^2,
  \end{equation}
  This shows that $m^2$ is even. \notemarkwavy{Hence}{因此} $m$ is even (if $m$ \notemarkwavy{were}{虚拟语气} odd, $m^2$ would be odd), and so $m^2$ is divisible by 4. It follows that the right side of \eqref{eq:2.2} is \notemarkwavy{divisible by}{被~... 整除} 4, so that $n^2$ is even, which implies that $n$ is even.

  The \notemarkwavy{assumption}{假定} that \eqref{eq:2.1} holds thus \notemarkwavy{leads to}{导致} the conclusion that both $m$ and $n$ are even, \notemarkwavy{contrary}{相反的} to our choice of $m$ and $n$. Hence \eqref{eq:2.2} is \notemarkwavy{impossible}{不可能的} for rational $p$.

  We now \notemarkwavy{examine}{研究} this situation \notemarkwavy{a little more closely}{更加仔细地}. Let $A$ be the set of all positive rationals $p$ such that $p^2<2$ and let $B$ \notemarkwavy{consist of}{由~...组成} all positive rationals $p$ such that $p^2>2$. We shall show that $A$ contains no largest number and $B$ contains no smallest.

  \notemarkwavy{More explicitly}{更加清晰地}, for every $p$ in $A$ we can find a rational $q$ in $A$ such that $p<q$, and for every $p$ in $B$ we can find a rational $q$ in $B$ such that $q<p$. To do this, we \notemarkwavy{associate}{关联, 联系} with each rational $p>0$ the number
  \begin{equation}\label{eq:2.3}
    q=p-\frac{p^2-2}{p+2}=\frac{2 p+2}{p+2}
  \end{equation}
  Then
  \begin{equation}\label{eq:2.4}
    q^2-2=\frac{2\left(p^2-2\right)}{(p+2)^2}
  \end{equation}
  
  If $p$ is in $A$ then $p^2-2<0$, \eqref{eq:2.3} \notemarkwavy{shows that}{表明} $q>p$, and \eqref{eq:2.4} shows that $q^2<2$. Thus $q$ is in $A$.

  If $p$ is in $B$ then $p^2-2>0$, \eqref{eq:2.3} shows that $0<q<p$, and \eqref{eq:2.4} shows that $q^2>2$. Thus $q$ is in $B$.
\end{theorem}

\begin{remark}
  The \underdotnotemark{purpose of}{... 的目标是} the above discussion has been to show that the rational number system has certain gaps, \underdotnotemark{in spite of the fact}{尽管事实如此} that between any two rationals there is another: If $r < s$ then $r < (r+s)/2 < s$. The real number system fills these gaps. This is the \underdotnotemark{principal}{最重要的, 首要的} reason for the \underdotnotemark{fundamental}{根本的, 基本的} role which it plays in analysis. 
\end{remark}